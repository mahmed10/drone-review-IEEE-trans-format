\section{Signal Processing and Data Management Framework for Drone}
\label{signalprocessingsection}
One of the major limitations in the field of machine learning is the limited space of an embedded system. So, it is important to store data more efficiently. One common technique is data compression while recording. 
A technique has been proposed that is motivated by \cite{hitomi2011video} imaging methodology to increase the frame rate of a normal off-the-shelf camera by using a silicon liquid crystal to compressively attain video-coded snapshots~\cite{kumar2018onboard}. Compression is performed out by using coded aperture imaging on a given hyperspectral datacube. For decompression, hey suggest a sparse recovery algorithm based on a deep neural network to rebuild the datacube from compressively coded snapshots. Whereas, this form of sparse data acquisition is usually decompressed using  sparse recovery algorithms such as orthogonal matching or iterative hard threshold, which is very slow process.

Another scheme to increase the working efficiency of the model is data augmentation. A method can generate sensor data of various kinds, such as camera images or Lidar point clouds~\cite{milz2018aerial}. The fundamental idea is to use a cGAN and the desired ground truth can be fed to any model as conditional input.
A two-stream CNN technique was presented that can predict the depth without using any depth sensor~\cite{kouris2018learning}. The network takes consecutive RGB image frames as input and returns the distance of any object from the drone in three directions. The CNN network is trained with a custom dataset where the actual distance from the object to the drone was collected using external HC-SR04 Ultrasonic Sensor and GP2Y0A60SZLF Analog IR Sensor mounted on the drone.