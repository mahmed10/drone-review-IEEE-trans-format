\section{Challenges}
\label{challengesection}
\textcolor{red}{
The rapid technological progress has attributed to a sharp rise in the popularity of drones recently as the evolving technology has made it easier than ever for anyone to fly drones without much training. This proliferation is not an unmixed blessing. Privacy and security\cite{custers2016flying} issues have become a major concern, threats of international and corporate espionage, terrorist sabotage, stalking and identity theft are becoming more prevalent.
}
% But more issues emerge with that accessibility. Privacy issues are high on the list of potential drone threats \cite{custers2016flying}. 
Drones can be used to spy on individuals, to track people persistently, and to collect large scale personal data. 
\textcolor{red}{
The data collection procedures  have come under scrutiny due to the lack of transparency;
the complexities of hardware and software (especially cloud) components in drones can make it difficult to trace the flow of information. Luckily, several national laws have been enacted to ensure legal protection for privacy; ensuring that the collection and processing of personal data should have a legitimate justification, and is subject to the principles of fair data processing.} \textcolor{orange}{Therefore, it is mandatory to ensure that the research work is being conducted, maintaining the personal privacy protocol, and doing no harm to the general people.}
% The rules for personal data processing are applied in several national laws,  
% Therefore, before start using a drone, we need to have a clear vision of drone application.

\textcolor{orange}{Another challenge that researchers may face during working with drones is hardware constraints.}
\textcolor{red}{
Although the sensing and navigation technology of drones has improved significantly over the years, flight endurance and payload capacity have remained relatively unchanged. Most off-the-shelf drones can only sustain a 15-30 minute flight before required to recharge of exchange the batteries. 
The flight time is inversely correlated with the payload capacity; a longer flight comes at the expense of fewer payloads or less on-board sensors. 
}
Besides this, networking is another factor that needs to be handled while working with drones.  \textcolor{orange}{Operating frequency, data transmission frequencies need to be chosen more carefully to prevent circumstances like the drone being operated by the controller of someone else.} 
\az{why?-answered}
Besides, functioning in an online network often means the possibility that the network will get hacked. \textcolor{orange}{Hackers may invade the data transmission channel. This downlink threat permits a hacker to retrieve videos, images, or other information that the drone is supposed to transmit only to the base station. Besides, they can take complete control of the drone by breaching the operating transmission signal. One approach commonly used by hackers is GPS spoofing~\cite{dronehacking}. Many of the drones fly with GPS data. If a hacker can feed false GPS coordination to the drone, then the hacker can drive them anywhere and steal the drone. Another way of hacking is taking advantage of unencrypted RFID signals. If the radio signal sent from the operator is unencrypted, then the hackers can easily decode the signal with a packet analyzer and hence take control over the drone. Signal jamming is also way applied by the hackers that leaves the drone no way to navigate.}
\az{need details -provided}
%Therefore, the firewall of any drone system should be strong. 
\textcolor{orange}{To prevent all these cyberattacks, some maintenance procedures need to be taken, e.g., updating the drone's firmware and firewall regularly, using a strong password, enabling ``return to home" mode whenever the drone loses connection, utilizing an encrypted communication signal, etc.}
\az{need specification -done}

\textcolor{orange}{To operate with a drone researchers may face some obstacles due to the future changes in drone flying rules. According to FAA (Federal Aviation Administration), anyone can fly a drone under 400 ft above the ground in unregulated airspace~\cite{FAAdrone} maintaining few guidelines. The number of drones operating in that below 400 ft airspace will be increased in the next 10-20 years~\cite{NASAdrone}. To provide guidelines for these drone operations, where air transport services from the FAA are not provided, FAA collaborating with NASA and other federal associate agencies is enabling a system, namely UAS Traffic Management (UTM)~\cite{kopardekar2014unmanned}. By providing services such as airspace architecture and dynamic configuration, dynamic geo-fencing, weather extremes, wind prevention, congestion management, terrain avoidance, trip planning, UTM's objective is to enable secure and efficient low-altitude airspace operational activities~\cite{dronetraffic}. So, researchers need to be updated with the revised rule by UTM and plan their research accordingly.}
\az{need either citation or some stats to support the claim
-done}