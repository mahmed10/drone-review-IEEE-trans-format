\section{Challenges}
\label{challengesection}
Drones are becoming more popular in recent days because of technological progress. High-tech systems keep evolving and make flying the drones much easier for us. But more issues emerge with that accessibility. Privacy issues are high on the list of potential drone threats \cite{custers2016flying}. Drones can be used to spy on individuals, to track people persistently, and to collect large scale personal data. The manner drones are being used to gather data, have minimal transparency. Several laws have been enforced to ensure legal protection for privacy. The rules for personal data processing are applied in several national laws, ensuring that the collection and processing of personal data should have a legitimate justification, and is subject to the principles of fair data processing. Therefore, before start using a drone, we need to have a clear vision of drone application.

Drones have several technological limitations. The main ones are limited flight endurance and payload capacity. Most of the common drones nowadays have maximum flight time from 15 to 30 minutes before exchanging or recharging batteries. As flight time of drone has an inverse relationship with payload capacity, one way to increase flight time is to reduce payload. But reducing payload means fewer sensors can be attached to the drone. Besides this, networking is another factor that needs to be handled while working with drones. Operating frequency, data transmission frequencies need to be chosen more carefully. Function in an online network often means the possibility that the network will get hacked. Therefore, the firewall of any drone system should be strong. Another task in the future regarding drones might be the development of the ultimate traffic management system. There is a high probability that drones will become familiar in our daily lives and any traffic control program is being built today, would be out of date.